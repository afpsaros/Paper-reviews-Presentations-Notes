\documentclass{article}
%%%%%%%%%%%%%%%%%%%%%%%%%%%%%%%%%%%%%%%%%%%%%%%%%%%%%%%%%%%%%%%%%%%%%%%%%%%%%%%%%%%%%%%%%%%%%%%%%%%%%%%%%
\usepackage{csquotes,xpatch}% recommended
\usepackage[backend=bibtex,
style=authoryear-comp,
sortcites=false,
maxbibnames=5,maxcitenames=2,
firstinits=true,
natbib=true,
]{biblatex}

\addbibresource{refs.bib}

% natbib = true: add comma between author and year
% firstinits: for first name initials in bibliography
\renewcommand{\postnotedelim}{ } % remove comma in post citation in autocite
%\addbibresource{refs.bib}
%%%%%%%%%%%%%%%%%%%%%%%%%%%%%%%%%%%%%%%%%%%%%%%%%%%%%%%%%%%%%%%%%%%%%%%%%%%%%%%%%%%%%%%%%%%%%%%%%%%%%%%%%

% Combine label and labelyear links
\xpatchbibmacro{cite}
{\usebibmacro{cite:label}%
	\setunit{\addspace}%
	\usebibmacro{cite:labelyear+extrayear}}
{\printtext[bibhyperref]{%
		\DeclareFieldAlias{bibhyperref}{default}%
		\usebibmacro{cite:label}%
		\setunit{\addspace}%
		\usebibmacro{cite:labelyear+extrayear}}}{}{}

% Include labelname in labelyear link
\xpatchbibmacro{cite}
{\printnames{labelname}%
	\setunit{\nameyeardelim}%
	\usebibmacro{cite:labelyear+extrayear}}
{\printtext[bibhyperref]{%
		\DeclareFieldAlias{bibhyperref}{default}%
		\printnames{labelname}%
		\setunit{\nameyeardelim}%
		\usebibmacro{cite:labelyear+extrayear}}}{}{}

% Access hyperref's citation link start/end commands
\makeatletter
\protected\def\blx@imc@biblinkstart{%
	\@ifnextchar[%]
	{\blx@biblinkstart}
	{\blx@biblinkstart[\abx@field@entrykey]}}
\def\blx@biblinkstart[#1]{%
	\blx@sfsave\hyper@natlinkstart{\the\c@refsection @#1}\blx@sfrest}
\protected\def\blx@imc@biblinkend{%
	\blx@sfsave\hyper@natlinkend\blx@sfrest}
\blx@regimcs{\biblinkstart \biblinkend}
\makeatother

\newbool{cbx:link}

% Include parentheses around labelyear in \textcite only in
% single citations without pre- and postnotes
\def\iflinkparens{%
	\ifboolexpr{ test {\ifnumequal{\value{multicitetotal}}{0}} and
		test {\ifnumequal{\value{citetotal}}{1}} and
		test {\iffieldundef{prenote}} and
		test {\iffieldundef{postnote}} }}

\xpatchbibmacro{textcite}
{\printnames{labelname}}
{\iflinkparens
	{\DeclareFieldAlias{bibhyperref}{default}%
		\global\booltrue{cbx:link}\biblinkstart%
		\printnames{labelname}}
	{\printtext[bibhyperref]{\printnames{labelname}}}}{}{}

\xpatchbibmacro{textcite}
{\usebibmacro{cite:label}}
{\iflinkparens
	{\DeclareFieldAlias{bibhyperref}{default}%
		\global\booltrue{cbx:link}\biblinkstart%
		\usebibmacro{cite:label}}
	{\usebibmacro{cite:label}}}{}{}

\xpretobibmacro{textcite:postnote}
{\ifbool{cbx:link}% patch 2.7+
	{\ifbool{cbx:parens}
		{\bibcloseparen\global\boolfalse{cbx:parens}}
		{}%
		\biblinkend\global\boolfalse{cbx:link}}
	{}}
{}
{\xpatchbibmacro{textcite}% patch earlier releases
	{\setunit{%
			\ifbool{cbx:parens}
			{\bibcloseparen\global\boolfalse{cbx:parens}}
			{}%
			\multicitedelim}}
	{\ifbool{cbx:link}
		{\ifbool{cbx:parens}
			{\bibcloseparen\global\boolfalse{cbx:parens}}
			{}%
			\biblinkend\global\boolfalse{cbx:link}}
		{}%
		\setunit{%
			\ifbool{cbx:parens}
			{\bibcloseparen\global\boolfalse{cbx:parens}}
			{}%
			\multicitedelim}}
	{}{}}
%%%%%%%%%%%%%%%%%%%%%%%%%%%%%%%%%%%%%%%%%%%%%%%%%%%%%%%%%%%%%%%%%%%%%%%%%%%%%%%%%%%%%%%%%%%%%%%%%%%%%%%%%
\DeclareNameAlias{sortname}{last-first} % last name first
\renewbibmacro{in:}{} % remove in: before journal

%%%%%%%%%%%%%%%%%%%%%%%%%%%%%%%%%%%%%%%%%%%%%%%%%%%%%%%%%%%%%%%%%%%%%%%%%%%%%%%%%%%%%%%%%%%%%%%%%%%%%%%%%
\usepackage{graphicx}
\usepackage{epstopdf} 
%%%%%%%%%%%%%%%%%%%%%%%%%%%%%%%%%%%%%%%%%%%%%%%%%%%%%%%%%%%%%%%%%%%%%%%%%%%%%%%%%%%%%%%%%%%%%%%%%%%%%%%%%
\usepackage{calrsfs}
\usepackage{physics}
\usepackage{mathtools}  
\usepackage{amsmath}
\usepackage{amssymb}
\usepackage{tabulary}
\usepackage{booktabs}
\usepackage{hyperref}
%%%%%%%%%%%%%%%%%%%%%%%%%%%%%%%%%%%%%%%%%%%%%%%%%%%%%%%%%%%%%%%%%%%%%%%%%%%%%%%%%%%%%%%%%%%%%%%%%%%%%%%%%
%\usepackage{chngcntr}
%\numberwithin{equation}{chapter}
%\counterwithin{figure}{chapter}
%%%%%%%%%%%%%%%%%%%%%%%%%%%%%%%%%%%%%%%%%%%%%%%%%%%%%%%%%%%%%%%%%%%%%%%%%%%%%%%%%%%%%%%%%%%%%%%%%%%%%%%%%
\setlength{\parindent}{2em}
\setlength{\parskip}{1em}

\linespread{1.6}
\usepackage{geometry}
\geometry{
	a4paper,
	total={134mm,225mm},
	left=38mm,
	top=35mm,
	headsep=.5in
}
\raggedbottom
%%%%%%%%%%%%%%%%%%%%%%%%%%%%%%%%%%%%%%%%%%%%%%%%%%%%%%%%%%%%%%%%%%%%%%%%%%%%%%%%%%%%%%%%%%%%%%%%%%%%%%%%%
\usepackage{blindtext}
\usepackage{ragged2e}
\usepackage{float}

\usepackage{epstopdf}
\usepackage{empheq} 

\usepackage{array}
\hypersetup{
	colorlinks
}
%%%%%%%%%%%%%%%%%%%%%%%%%%%%%%%%%%%%%%%%%%%%%%%%%%%%%%%%%%%%%%%%%%%%%%%%%%%%%%%%%%%%%%%%%%%%%%%%%%%%%%%%%
\usepackage{graphics}
\graphicspath{ {figures/} }
\renewcommand{\listfigurename}{List of figures}

\usepackage[labelfont=bf,justification=justified,singlelinecheck=false]{caption}
\captionsetup[figure]{name=Fig. ,labelsep=period}
\captionsetup[table]{labelsep=period}
\captionsetup[figure]{labelfont={bf},labelformat={default},labelsep=period,name={Fig.}}
%%%%%%%%%%%%%%%%%%%%%%%%%%%%%%%%%%%%%%%%%%%%%%%%%%%%%%%%%%%%%%%%%%%%%%%%%%%%%%%%%%%%%%%%%%%%%%%%%%%%%%%%%
\usepackage{array}
\usepackage{longtable}
\usepackage{xcolor}

\usepackage{comment}

\usepackage{enumitem}

\usepackage{wrapfig}
%%%%%%%%%%%%%%%%%%%%%%%%%%%%%%%%%%%%%%%%%%%%%%%%%%%%%%%%%%%%%%%%%%%%%%%%%%%%%%%%%%%%%%%%%%%%%%%%%%%%%%%%%
\usepackage{titlesec}

\titlespacing*{\section}
{0pt}{1ex plus .5ex minus .2ex}{.5ex plus .2ex}
\titlespacing*{\subsection}
{0pt}{0.5ex plus .5ex minus .2ex}{.5ex plus .2ex}
%\titlespacing*{\subparagraph}
%{0pt}{2.5ex plus 1ex minus .2ex}{1.3ex plus .2ex}

\setcounter{secnumdepth}{4}
\setcounter{tocdepth}{4}

\newcommand{\hsp}{\hspace{5pt}}

\titleformat{\section}[block]{\bfseries\large}{\thesection}{1em}{}
\titleformat{\subsection}[block]{\bfseries\itshape}{\thesubsection}{1em}{}


%\titleformat{\subsubsection}
%{\normalfont\normalsize\itshape}{\thesubsubsection}{1em}{}
%\titleformat{\subparagraph}[runin]
%{\itshape\normalsize}{\thesubparagraph}{1em}{}

%%%%%%%%%%%%%%%%%%%%%%%%%%%%%%%%%%%%%%%%%%%%%%%%%%%%%%%%%%%%%%%%%%%%%%%%%%%%%%%%%%%%%%%%%%%%%%%%%%%%%%%%%
\usepackage{subcaption}
\usepackage{bbm}
\usepackage{tabularx}
%%%%%%%%%%%%%%%%%%%%%%%%%%%%%%%%%%%%%%%%%%%%%%%%%%%%%%%%%%%%%%%%%%%%%%%%%%%%%%%%%%%%%%%%%%%%%%%%%%%%%%%%%
\definecolor{mycolor}{RGB}{207,42,40}
\AtBeginDocument{\hypersetup{citecolor=violet, linkcolor = mycolor}}

\usepackage{indentfirst}


%%%%%%%%%%%%%%%%%%%%%%%%%%%%%%%%%%%%%%%%%%%%%%%%%%%%%%%%%%%%%%%%%%%%%%%%%%%%%%%%%%%%%%%%%%%%%%%%%%%%%%%%

\DeclareMathAlphabet{\pazocal}{OMS}{zplm}{m}{n}
\newcommand{\bw}{\boldsymbol{w}}
\newcommand{\bp}{\boldsymbol{p}}
\newcommand{\bth}{\boldsymbol{\theta}}
\newcommand{\bA}{\boldsymbol{A}}
\newcommand{\cH}{\pazocal{H}}
\newcommand{\cN}{\pazocal{N}}
\newcommand{\cP}{\pazocal{P}}
\newcommand{\cD}{\pazocal{D}}
\newcommand{\cO}{\pazocal{O}}
\newcommand{\cL}{\pazocal{L}}


\setcounter{tocdepth}{3}
\begin{document}
	
	\sloppy
	
%%%%%%%%%%%%%%%%%%%%%%%%%%%%%%%%%%%%%%%%%%%%%%%%%%%%%%%%%%%%%%%%%%%%%%%%%%%%%%%%%%%%%%%%%%%%%%%%%%%%%%%%%
	\begin{center}	
		\Large
		\textbf{Dropout as variational inference}\\
		\large
		Apostolos Psaros\\	
%		\today
%		July 10, 2020
	\end{center}
	\vskip 0.25in
	
%%%%%%%%%%%%%%%%%%%%%%%%%%%%%%%%%%%%%%%%%%%%%%%%%%%%%%%%%%%%%%%%%%%%%%%%%%%%%%%%%%%%%%%%%%%%%%%%%%%%%%%%%
%{\footnotesize
%\setlength{\parskip}{0.1em}
%\linespread{0.1}
%\tableofcontents
%\newpage}


%\setlength{\parskip}{1em}
%\linespread{1.6}
\section{Forward pass with dropout}
Consider a NN with a single hidden layer. 
The hidden layer weights and biases are $\boldsymbol{m}_1$ and $\boldsymbol{b}$, respectively.
These are vectors of size $K$, the width of the layer.
The output weights are $\boldsymbol{m}_2$, also of size $K$. 
For doing a forward pass with dropout we sample a vector $\hat{\boldsymbol{\epsilon}} \sim p(\boldsymbol{\epsilon})$ of dimension $K$. 
The elements of $\hat{\boldsymbol{\epsilon}}$ take value $0$ with probability $0 \leq p \leq 1$. 
Given the output of the hidden layer $\boldsymbol{h} = \sigma(x\boldsymbol{m}_1+\boldsymbol{b})$, we set a proportion of it to zero, i.e., $\hat{\boldsymbol{h}} = \boldsymbol{h}\odot \hat{\boldsymbol{\epsilon}}$. 
Finally, $f_{\cH}(x_i;\bth, \hat{\boldsymbol{\epsilon}})= \boldsymbol{m}_2^T\hat{\boldsymbol{h}}$, where $\bth = \{\boldsymbol{m}_1, \boldsymbol{m}_2, \boldsymbol{b}\}$.


\section{Training with dropout}
Training involves obtaining $\bth$ by minimizing 
\begin{equation}
	\cL_{dropout}(\bth) = \sum_{i = 1}^{N} \frac{1}{2}|y_i - f_{\cH}(x_i;\bth, \boldsymbol{\epsilon})|^2 + Reg(\bth)
\end{equation}
where $\boldsymbol{\epsilon}$ is a random variable as described above and $Reg(\bth)$ is a regularization term. 
In each iteration, for obtaining the gradient of $\cL_{dropout}$ we use a sample $\hat{\cL}_{dropout}$ given as 
\begin{equation}
\hat{\cL}_{dropout}(\bth) = \sum_{i = 1}^{N} \frac{1}{2}|y_i - f_{\cH}(x_i;\bth, \hat{\boldsymbol{\epsilon}})|^2 + Reg(\bth)
\end{equation}

\section{Equivalence with variational inference}
Note that 
\begin{equation} \label{}
\begin{split}
f_{\cH}(x_i;\bth, \hat{\boldsymbol{\epsilon}}) &= \boldsymbol{m}_2^T\hat{\boldsymbol{h}}\\
& = \boldsymbol{m}_2^T diag(\hat{\boldsymbol{\epsilon}})\boldsymbol{h}\\
& = \hat{\boldsymbol{w}}_2^T\boldsymbol{h}\\
& = f_{\cH}(x_i;\hat{\bw})
\end{split}
\end{equation}
where $\hat{\bw} = \{\boldsymbol{m}_1, \hat{\boldsymbol{w}}_2, \boldsymbol{b}\}$ is a sample of the random variable $\bw = \{\boldsymbol{m}_1, \boldsymbol{w}_2, \boldsymbol{b}\}$, with $\boldsymbol{w}_2 = \boldsymbol{m}_2\odot \boldsymbol{\epsilon}$ and $\hat{\boldsymbol{w}}_2 = \boldsymbol{m}_2\odot \hat{\boldsymbol{\epsilon}}$. 
We can therefore transfer the uncertainty from the feature space to the model weights. 
Therefore, the training loss in each iteration is 
\begin{equation}
\hat{\cL}_{dropout}(\bth) = \sum_{i = 1}^{N} \frac{1}{2}|y_i - f_{\cH}(x_i;\hat{\bw})|^2 + Reg(\bth)
\end{equation}
which can also be written as
\begin{equation}\label{dropout_loss}
\hat{\cL}_{dropout}(\bth) = -\beta\sum_{i = 1}^{N} \log p(y_i|\hat{\bw},x_i,\cH) + Reg(\bth) 
\end{equation}

If we write $\bw=\boldsymbol{g}(\bth,\boldsymbol{\epsilon}) = \{\boldsymbol{m}_1, \boldsymbol{m}_2\odot \boldsymbol{\epsilon}, \boldsymbol{b}\}$, Eq.~\eqref{dropout_loss} becomes
\begin{equation}\label{}
\hat{\cL}_{dropout}(\bth) = -\beta\sum_{i = 1}^{N} \log p(y_i|\boldsymbol{g}(\bth,\hat{\boldsymbol{\epsilon}}),x_i,\cH) + Reg(\bth)
\end{equation}
Recall that the loss in Bayes by backprop is
\begin{equation}\label{update}
\hat{\cL}_{BBB}(\bth)= -\sum_{i=1}^{N} \log p(y_i|\boldsymbol{g}(\bth, \hat{\boldsymbol{\epsilon}}_i),x_i,\cH) + KL\left(q_{\bth}(\bw)||p(\bw|\cH)\right)
\end{equation}

\section{Summary}
Training with dropout is equivalent to Bayes by backprop 
\begin{enumerate}
	\item with a difference in scale in the summation term
	\item with reparametrization $\boldsymbol{g}(\bth,\boldsymbol{\epsilon}) = \{\boldsymbol{m}_1, \boldsymbol{m}_2\odot \boldsymbol{\epsilon}, \boldsymbol{b}\}$
	\item with prior $p(\bw|\cH)$ and approximating distribution $q_{\bth}(\bw)$ such that $KL\left(q_{\bth}(\bw)||p(\bw|\cH)\right) = Reg(\bth) $.
\end{enumerate}

Two more notes:
\begin{enumerate}
	\item Other stochastic regularization techniques can be recovered with different reparametrizations and  $\boldsymbol{g}(\bth,\boldsymbol{\epsilon})$
	\item after training with dropout the NN can be used exactly as a BNN (MC dropout).
\end{enumerate}

Overall, optimizing any NN with dropout is equivalent to \textit{a form} of variational inference. 

%%%%%%%%%%%%%%%%%%%%%%%%%%%%%%%%%%%%%%%%%%%%%%%%%%%%%%%%%%%%%%%%%%%%%%%%%%%%%%%%%%%%%%%%%%%%%%%%%%%%%%%%%	

%\section*{Appendix}
\newpage	
\printbibliography[heading=bibintoc,title={References}]
	
\end{document}